% Created 2017-02-28 Tue 13:54
\documentclass[presentation]{beamer}
% \usepackage[utf8]{inputenc}
% \usepackage[T1]{fontenc}
% \usepackage{fixltx2e}
\usepackage{graphicx}
\usepackage{longtable}
\usepackage{float}
\usepackage{wrapfig}
\usepackage{rotating}
\usepackage[normalem]{ulem}
\usepackage{amsmath}
\usepackage{textcomp}
\usepackage{marvosym}
\usepackage{wasysym}
\usepackage{amssymb}
\usepackage{hyperref}

\tolerance=1000
\usetheme{CambridgeUS}
\usecolortheme{seagull}
\setbeamertemplate{navigation symbols}{}%remove navigation symbols
\setbeamertemplate{itemize items}{\color{red}$\CIRCLE$}%[circle]

\author[Boribong, Jalihal]{Amogh Jalihal, Brittany Boribong}
\date{\today}
\title{Gene-for-Gene Models: Controversies and Results}
\hypersetup{
  pdfkeywords={},
  pdfsubject={},
  pdfcreator={Emacs 24.5.1 (Org mode 8.2.10)}}



\begin{document}
\maketitle
\begin{frame}[label=sec-1]{History of \emph{gene-for-gene} systems}
  \begin{itemize}
  \item Significance: Plant-pathogen interactions
  \item Agriculture post-``Dirty Thirties''
  \item Flor proposed a gene-for-gene hypothesis \footnote{\tiny{Flor, 1942 \cite{flor1942complementary}}}
  \end{itemize}
    \begin{quotation}
      ...for each gene determining resistance in flax (\emph{Linum usitatissimum}) there was a specific and related gene determining pathogenicity in the rust fungus (\emph{Melampsora lini}). 
    \end{quotation}
    \begin{itemize}
    \item Biological basis of the gene-for-gene hypothesis \footnote{\tiny{van der Plank, 1968 \cite{vanderplank1968disease}}}
    \end{itemize}
\end{frame}
\begin{frame}[label=sec-2]{Enter Population Genetics}
\begin{table}[h!]
\centering
\begin{tabular}{|c|c|}
\hline
\textbf{Term} & \textbf{Definition} \\
\hline
$p$ & Frequency of resistant allele in hosts \\
$q$ & Frequency of susceptible allele in hosts \\
$m$ & Frequency of virulent allele in pathogens\\
$n$ & Frequency of avirulent allele in pathogens\\
$k$ & Cost of expressing virulences\\
$t$ & Loss of fitness of avirulent pathogen on resistant host\\
$a$ & Fitness gain of expressing virulence in pathogen\\
$c$ & Cost of resistance\\
$\mu$ & Mutation rate\\
\hline
\end{tabular}
\caption{List of variables and parameters used}
\label{tab:parlist}
\end{table}
\end{frame}
\begin{frame}{Enter Population Genetics}
Relative fitnesses of pathogen alleles on the varying host phenotypes
\begin{center}
\begin{tabular}{c|c|c}
& \textbf{$r$} & \textbf{$R$}\\
\hline
$v$ & 1 & $1-t$\\
$V$ & $1-k$ & $1-k+a$\\
\end{tabular}
\end{center}
Pathogen alleles: $v$ = avirulent, $V$ = virulent \\
Host alleles: $r$ = susceptible, $R$ = resistant \\
\pause
Fitness ($W$) of the two pathogenic alleles in the next generation
\begin{align}
W_V&=q^2(1-k)+(1-q^2)(1-k+a) \label{eq:fitV} \\
W_v&=q^2 + (1-q^2)(1-t) \label{eq:fitv}
\end{align}
$q$, $p=1-q$ represent the population frequencies of the susceptible and resistant host alleles, respectively
\end{frame}
\begin{frame}{Enter Population Genetics}
The frequency of the virulent allele in the next generation can be expressed in terms of the fraction of hosts infected by each pathogenic variant
\begin{align*}
n_{i+1}=\frac{n_iW_V}{n_iW_V+m_iW_v}
\end{align*}
where $n$ is the frequency of $V$, and $m$ is the frequency of $v$.
\end{frame}
\begin{frame}[label=PathogenDerivation]{Enter Population Genetics}
\begin{align}
n_{i+1} &= \frac{n_i[1-k+(1-q_i^2)a]}{1-(1-q_i^2)t+n_i[(1-q_i^2)(a+t)-k]} \label{eq:Leonard94-1} \\
p_{i+1} &= \frac{p_i[1-c-s(1-t)+n_{i+1}s(k-a-t)]}{1-s+n_{i+1}ks+(1-q_i^2)[ts-c-n_{i+1}s(a+t)]} \label{eq:Leonard94-2}
\end{align}
\vfill{}
\hyperlink{HostDerivation}{\beamerbutton{Derivation of Host Population Model}}
\end{frame}
\begin{frame}[label=sec-3]{Controversy regarding stability: Sedcole, 1978}
\end{frame}
\begin{frame}[label=sec-4]{Controversy regarding stability: Leonard and Czochor,1978}
\end{frame}
\begin{frame}[label=sec-5]{Controversy regarding stability: Fleming, 1978}
\end{frame}
\begin{frame}[label=sec-6]{Revisiting dynamics: Leonard, 1994}
\end{frame}
\begin{frame}[label=sec-7]{The Problem with Leonard: Kirby and Burdon, 1997}
\end{frame}
\begin{frame}[label=sec-8]{Our Problems with Kirby and Burdon, 1997}
\end{frame}
\begin{frame}{References}
\bibliographystyle{unsrt}
\bibliography{references}
\end{frame}
\appendix
\begin{frame}[label=HostDerivation]{Derivation of Host Population Model}
\hyperlink{PathogenDerivation}{\beamerbutton{Main}}
\end{frame}
% Emacs 24.5.1 (Org mode 8.2.10)
\end{document}